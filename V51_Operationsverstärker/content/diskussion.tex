\section{Diskussion}
\label{sec:diskussion}
In diesem Teil des Protokolls werden die Ergebnisse mit den theoretischen Werten verglichen um mögliche Fehlerquellen erläutert, die zu Abweichungen zwischen Theorie und Experiment führen.
\subscetion{invertierter Linearverstärker}
Zunächst wird für den Linearverstärker die Abweichung der Verstärkung berechnet, diese ist in Tabelle \ref{tab:lin_disk} zu sehen.
Die Abweichung liegen alle unter $2\,\%$ und sind damit im experimentellen Rahmen.
\begin{table}
    \centering
    \begin{tabular}{cccc}
        \toprule
        &Messung 1 & Messung 2 & Messung 3\\
        \midrule
        Mittelwert $\bar{V}'$ & $\SI{93.600(2100)}{}$ &  $\SI{2.080(1)}{} $&  $\SI{4.680(50)}{} $ \\
        theoretische Verstärkung $V_\text{theo}'$ & 100 & 2.126 & 4.680 \\
        Abweichung$\,\%$ & $\SI{6.4(21)}{} $& $\SI{2.2(1)}{}$ & $\SI{0.0(10)}{}$ \\
        \botoomrule
    \end{tabular}
    \caption{Die Abweichung der theoretischen Verstärkung zur experimentellen Verstärkung.}
    \label{tab:lin_disk}
\end{table}
Die Verstärkung der invertierenden Linearverstärkers zeigt das erwartete Plateu für niedrige Frequenz.
Ab einer Grenzfrequenz, die mit der theoretischen Grenzfrequenz übereinstimmt, fällt die Verstärkung exponentiell bis auf 0 ab.
\subsection{Umkehr Integrator und invertierender-Differenzierer}
Der Umkehrintegrator und der invertierender-Differenzierer bilden wie erwartet die Stammfunktion beziehungsweise die Ableitung des Eingangssignal, was in den Abbildung \ref{fig:umkehr_oszi} und \ref{fig:dif_oszi} zu erkennen ist.
Die Ausgangsspannung weist das ewartete Verhalten auf und eine Ausgleichsgerade konnte gut an die Messwerte gelegt werden.
\subsection{Schmitt-Trigger}
Wie erwartet springt die Ausgangsspannung des Schmitt-Triggers ab einer Grenzamplitude auf $\approx \SI{30}{\V}$.
Die Abweichung der experimentellen Grenzamplitude $U_\text{Schmitt} &= \SI{3.26}{\V}$ zu der theoretischen $U_\text{Schmitt,theo} &= \SI{2.91(5)}{\V}$ beträgt $Abw = \SI{12.03(19)}{}\%$.
Der Schmitt-Trigger hat damit wie erwartet funktioniert und weißt nur gewisse Abweichung zwischen Experiment und Theorie auf.
Diese kann mit den schlechten Kontakten der Steckplatine begründert werden, die zu einem höheren Widerstand geführt haben können.
\subsection{Signalgenerator}
Der Generator hat das erwartet Dreiecksspannungs Signal ausgegeben, allerdings hatte dieses mit $U_\text{a,2} &= \SI{5.20(5)}{\V}$ eine höhere Spannung als erwatet $U_\text{a,2,theo} &= \SI{2.77(5)}{\V}$.
Die Abweichung des experimentellen Wertes zum theoretischen beträgt $Abw = \SI{87.73(34)}{}\%$.
Es ist wahrscheinlich, dass auch hier die Steckplatine für einen höheren Widerstand gesorgt hat, weswegen das Ausgangssignal verfälscht wurde.
\\\\
Die Schaltungen könnten durch gelötete Kontakt verbessert werden, da diese weniger Fehleranfällig sind und nur einen sehr kleinen Widerstand haben.
Dies würde natürlich zu einem größeren Aufwand führen.
Zudem schienen die Operationsverstärker teilweise kaputt zu sein, was ebenfalls zu Fehlern geführt haben könnte.

\section{Auswertung}
\label{sec:auswertung}
In diesem Teil werden die Messwerte der verschiedenen Schlatungen vorgestellt und erläutert.
Dabei wird in der Reihenfolge der Durchführung vorgegangen.
\subsection{Fehlerrechung}

\subsection{invertierter Linearverstärker}
Als erste Schaltung wird der invertierter Linearverstärker aufgebaut.
Der Aufbau ist in Abschnitt \section{sec:durchfuehrung} im Detail erklärt.
Für die Schaltung werden drei Messungen durchgeführt.
Dabei werden die Widerstände ausgetauscht um unterschiedliche Spannungsverstärkungen zu erzielen.

Die genutzten Widerstände der verschiedenen Messungen sind in Tabelle \ref{tab:wider_inv_lin} zu sehen.
\begin{table}
    \centering
    \begin{tabular}{ccc}
        \toprule
        Messung & $R_1 \, / \, k\Omega $ & $R_2 \, / \,  k\Omega $ \\
        \midrule 
        1 & 1 & 100 \\
        2 & 47 & 100 \\
        3 & 47 & 220 \\
        \bottomrule
    \end{tabular}
    \caption{Die genutzen Widerstände für den invertierter Linearverstärker für drei Messdurchgänge.}
    \label{tab:wider_inv_lin}
\end{table}
Bei den Messdurchgänge werden wie in Abschnitt \ref{sec:durchfuehrung} beschrieben, für verschiedenen Frequenzen die Phase zwischen Eingangs und Ausgangsspannung, sowie die Höhe der Ausgangsspannung gemessen.
Danach wird aus dem Quotienten der Eingangsspannung und Ausgangsspannung die Verstärkung $V^{,}$ berechnet.
Die Verstärkung ist in Form eines doppelt logarithmischen Plots in Abbildung \ref{fig:inv_lin} zu sehen.
\begin{figure}
    \centering
    \begin{subfigure}{0.49\linewidth}%
        \includegraphics[width=\textwidth]{build/inv_lin_1.pdf}
        \subcaption{Messwerte der Messung 1.}
    \end{subfigure}
    \hfill
    \begin{subfigure}{0.49\linewidth}%
        \includegraphics[width=\textwidth]{build/inv_lin_2.pdf}
        \subcaption{Messwerte der Messung 2.}
    \end{subfigure}\\
    \begin{subfigure}{0.49\linewidth}%
        \includegraphics[width=\textwidth]{build/inv_lin_3.pdf}
        \subcaption{Messwerte der Messung 3.}
    \end{subfigure}
    \caption{Die Messwerte der Verstärkung durch den invertierten Linearverstärkers. Es wird die Verstärkung gegen die Frequenz der Eingangsspannung geplottet.
    Dies geschieht in einer Plot mit doppelt logarithmischer Achse. Ein Fit wird für den abfallenden Bereich angefertigt, welcher als blaue Linie zu erkennen ist.}
    \label{fig:inv_lin}
\end{figure}
Die Ausgleichfunktion für die Messwerte wird mithilfe der Funktion
\begin{equation}
    Fit(f) = a f^b
\end{equation}
durchgeführt.
Die Ausgleichrechung ergibt für die Parameter $a$ und $b$ der verschiedenen Messreiehen die Widerstände
\begin{align}

    Fit_1(f) &= \SI{6.9(28)e5}{\frac{\V}{\Hz}} f^{-\SI{1.02(4)}} \\
    Fit_2(f) &= \SI{5.7e8}{\frac{\V}{\Hz}} f^{-1.51}\\
    Fit_3(f) &= \SI{1.2e4}{\frac{\V}{\Hz}} f^{-0.68}

\end{align}

Aufgrund der doppelt logarithmischen Achsen und der geringen Anzahl von Messwerten im abfallenden Bereich, könne für Messung 2 und 3 durch die Funktion 'curve_fit' des python Pakets 'scipy' \cite{scipy} keine Unsicherheiten bestimmt werden.
Für den Plateubereich, welcher für geringe Frequenzen im invertierten Linearverstärker erkennbar ist wird der Mittelwert der Verstärkung berechnet.
Bei der Berechnung der Mittelwerte werden dabei unterschiedlich viele Messwerte genutzt, je nachdem wie lang das Plateu bei der jeweiligen Konfiguration ist.
In Tabelle \ref{tab:inv_lin_mittel} sind die berechneten Mittelwerte aufgetragen. 
In der Tabelle sind zudem die Anzahl der genutzten Messpunkte der jeweiligen Messung, sowie die theoretische Verstärkung zu sehen.
Die theoretische Verstärkung wird dabei nach Gleichung \eqref{eq:verstaerkung} berechnet. 
\begin{table}
    \centering
    \begin{tabular}{ccc}
        \toprule
        Messung & Anzahl Messpunkte & Mittelwert $\bar{V^{,}} $ & theoretische Verstärkung $V_\text{theo}^{,}$ \\
        \midrule
        1 & 4 & $\SI{93.600(2100)}$ & 100\\
        2 & 8 & $\SI{2.080(0)} $& 2.126 \\
        3 & 7 & $\SI{4.680(50)} $& 4.680 \\
        \bottomrule 
    \end{tabular}
    \label{tab:inv_lin_mittel}
\end{table} 
Abgesehen von der Ausgangsspannung, aus der die Verstärkung berechnet wird, wird auch die Phase zwischen Eingangsspannung und Ausgangsspannung gemessen.
Diese wird anschließend gegen die Frequenz aufgetragen.
Die resultierenden Plots sind in Abbildung \ref{fig:phase} zu sehen.
\begin{figure}
    \centering
    \begin{subfigure}{0.49\linewidth}%
        \includegraphics[width=\textwidth]{build/inv_phase1.pdf}
        \subcaption{Messwerte der Messung 1.}
    \end{subfigure}
    \hfill
    \begin{subfigure}{0.49\linewidth}%
        \includegraphics[width=\textwidth]{build/inv_phase2.pdf}
        \subcaption{Messwerte der Messung 2.}
    \end{subfigure}\\
    \begin{subfigure}{0.49\linewidth}%
        \includegraphics[width=\textwidth]{build/inv_phase3.pdf}
        \subcaption{Messwerte der Messung 3.}
    \end{subfigure}
    \caption{Die Messwerte der Phase zwischen Eingangsspannung und Ausgangsspannung beim invertierten Linearverstärkers. Die Phase wird gegen die Frequenz der Eingangsspannung geplottet.
    Dies geschieht in einer Plot mit logarithmischer y-Achse.}
    \label{fig:phase}
\end{figure}

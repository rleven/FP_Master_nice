\newpage
\section{Diskussion}
\label{sec:diskussion}
Bei der Justage des Interferometers konnte ein maximaler Kontrast von $0.938$ erreicht werden.
Dabei waren auf dem Abbildungsschirm des Interferzmusters keine Streifen zu sehen, was von einer guten Justage zeugt.\\\\
Im weiteren Verlauf des Versuchs wird der Brechungsindex bestimmt.
Der bestimmte Wert von 
\begin{align}
    n_\text{Glas} &= \SI{1.556(17)}{}
\end{align}
kann nicht zuverlässig mit einem Literaturwert verglichen werden, da nicht bekannt ist welches Glas im Versuch genutzt wird.
So beträgt die Abweichung vom Brechungsindex von Silizium Oxid, $n_{\text{SiO}_2} = 1.475$ rund $\SI{2.9(5)}{} \,\%$ \cite{Teschner2019}.
Zu Calcium Oxid $n_\text{CaO} = 1.730$ hingegen $\SI{5.3(5)}{}\,\%$.
Falls in dem Versuch eine Mischung aus Gläsern genutzt wird, könnte die Abweichung zum Literaturwert noch geringer sein.\\
Bessere Ergebnis könnten durch die Drehmethode erreicht werden, wenn der mögliche Drehwinkel größer wäre.
So wären kleine Schwankungen zwischen Endpunkt und Ausgangspunkt der Drehung weniger beeinträchtigend.
Außerdem könnte die Drehung automatisiert werden, sodass die Drehung vollkommen gleichmäßig stattfindet.
Dadurch könnte ein versehentliches zurück drehen vermieden werden.\\\\
Im letzten Versuchsteil wird der Brechungsindex von Raumluft bestimmt.
Der berechnete Brechungsindex bei Normaldruck ist
\begin{align*}
    \bar{n}_\text{Luft} &= \SI{1.0002681(07)}{}\,.
\end{align*}
Die Abweichung vom Literaturwert ${n_\text{Luft,Lit} = \SI{1.000292}{}}$ \cite{Dem2} beläuft sich so auf ${\SI{1.194(33)e-3}{\percent}}$.
Die Methode zur Bestimmung der Brechungsindex von Gasen scheint sehr genau zu sein.
Dennoch könnte auch hier die zuvor des Gases automatisiert werden, wodurch ein kontinuirlicher Gasfluss erreicht werden könnte.
\newpage
\section{Auswertung}
\label{sec:auswertung}

Dieser Teil des Protokolls beschäftigt sich mit der Auswertung der zuvor beschriebenen Versuchsteile.
Dabei soll der Brechungsindex von Glas, sowie von Raumluft bestimmt werden.\\\\
Wie im Abschnitt \ref{sec:durchfuehrung} beschrieben wird, soll zunächst der größte mögliche Kontrast bestimmt werden.
Dafür wird der Polfilter in $\SI{10}{\degree}$ Schritten rotiert.
Bei jedem Schritt werden die Intensitäten der beiden zueinander gegenseitig polarisierten Laserstrahlen gemessen.
Aus den Intensitäten wird anschließend der Kontrast, durch Gleichung \eqref{eq:kontrast} berechnet.
Die Messwerte werden danach gegeneinader aufgetragen, das Ergebnis ist in Abbildung \ref{fig:Kontrast} zu sehen.
\begin{figure}
    \centering
    \includegraphics[width=\textwidth]{build/Kontrast.pdf}
    \caption{Der berechnete Kontrast zu den Einestellungen des Polfilter in einem $\SI{180}{\degree}$ Bereich.
    Der maximal erreicht Kontrast wird mit einem roten Kreuz markiert.
    Es wird zudem ein Fit für die Messwerte angefertigt.}
    \label{fig:Kontrast}
\end{figure}
Der maximal erreichte Kontrast wird mit einem roten Kreuz markiert.
Dieser liegt bei einem Winkel von $\SI{50}{\degree}$, welches einem Kontrast von $0.938$ entspricht.
Alle weiteren Messwerte werden mit dieser Polfilter Einstellung vorgenommen.
Zudem wird mithilfe von Gleichung \eqref{eq:intense} ein Fit für die Messwerte angefertigt.
Dabei wird die Gleichung in der Form 
\begin{equation*}
    Kontrast =  A\cdot |(\cos(\phi)\cdot \sin(\phi))|
\end{equation*}
genutzt.
Der zu bestimmende Parameter $A$ wird mithilfe der Python Funktion \textit{curvefit} \cite{scipy} betsimmt.
Dieser ergibt sich zu 
\begin{align*}
    A = \SI{1.82(4)}{}
\end{align*}
wobei die Unsicherheit beim Erstellen des Plots vernachlässigt wird.
\subsection{Brechungsindex von Glas}
Um nun den Brechungsindex von Glas zu bestimmen werden die zwei Glasplättchen die im Strahlengang stehen zu einander verkippt.
Anders als zuvor stehen sie also nicht so, dass immer der maximale beziehungsweise minimale Kontrast gemessen wird.
Die Verkippung gelingt durch die Montierung der Glasplättchen auf einem Rotationshalter.
Dieser bewegt sich von $(0-10)\,\si{\degree}$.
Es wird nun die Anzahl an Maxima Durchläufen des Kontrasts gemessen.
Die Messung wird zehn mal wiederholt, sodass sich die Durchlaufsanzahlen der Tabelle \ref{tab:n_glas} ergeben.
Aus der Anzahl der Durchläufe kann nun der Brechungsindex von Glas mithilfe von Gleichung \eqref{eq:brechungsindex} bestimmt werden.
Dabei wird der Brechungsindex für jede Messung erst einzeln berechnet, wodurch sich die Brechungsindizes der Tabelle \ref{tab:n_glas} ergeben.
\begin{table}
    \hspace*{-1cm}
    \begin{tabular}{ccccccccccc}
        \toprule 
        Messung & 1&2&3&4&5&6&7&8&9&10\\
        \midrule
        Anzahl &34 & 34& 35& 35& 33& 35& 35& 34& 35& 34\\
        Brechungsindex &1.546&1.546&1.571&1.571&1.521&1.571& 1.5714&1.5462&1.571& 1.546\\
        \bottomrule
    \end{tabular}
    \caption{Die gemessenen Maximadurchläufe des Kontrast wenn die Glasplättchen um $\SI{10}{\degree}$ verdreht werden.
    Zudem werden die aus den Maximadurchläufen berechneten Brechungsindizes aufgezeigt.}
    \label{tab:n_glas}
\end{table}
\FloatBarrier\noindent
Anschließend werden die Brechungsindizes gemittelt, wodurch sich der Wert 
\begin{align*}
    n_\text{Glas} &= \SI{1.556(17)}{}
\end{align*}
ergibt.
\subsection{Brechungsindex von Raumluft}
Es wird nun der Glasplättchenhalter aus dem Versuchsaufbau ausgebaut und durch eine Gaszelle ersetzt.
Diese steht so, dass sie nur von einen Strahl durchlaufen wird.
In der Gaszelle befindet sich zunächst ein niedriger Gasdruck, welcher durch langsames hinzugeben von Raumluft bis auf $~\SI{1}{\bar}$ gebracht wird.
Während des Befüllen der Gaszelle wird die Anzahl der Maximadurchläufe des Kontrasts des Interferometers gemessen.
Die Anzahl wird dabei alle $\SI{50}{\milli\bar}$ aufgenommen.
Die Messung wird drei Mal wiederholt, sodass sich die Messwerte in Tabelle \ref{tab:gas_counts} ergeben.
\begin{landscape}
\begin{table}
    \centering
    \sisetup{table-format=1.7}
    \begin{tabular}{S[table-format=3] S[table-format=2] S @{${}\pm{}$} S S[table-format=2] S @{${}\pm{}$} S S[table-format=2] S @{${}\pm{}$} S}
        \toprule
         & \multicolumn{3}{c}{Messung 1} & \multicolumn{3}{c}{Messung 2} & \multicolumn{3}{c}{Messung 3}\\
        \cmidrule(lr){2-4}\cmidrule(lr){5-7}\cmidrule(lr){8-10}
        {Druck$ /\,\si{\milli\bar}$} & {Messung 1} & {$n_1$} & {Unsicherheit} &  {Messung 2} & {$n_2 $} & {Unsicherheit} & {Messung 3} & {$n_3$} & {Unsicherheit} \\
        \midrule 
        50 &     2&  1.0000126 & 0.0000018   &  2 &  1.0000126 & 0.0000012 &   2 &    1.0000126 & 0.0000012 \\                         
        100&     4&  1.0000253 & 0.0000011   &  4 &  1.0000253 & 0.0000025 &   4 &    1.0000253 & 0.0000025 \\
        150&     6&  1.0000379 & 0.0000019   &  6 &  1.0000379 & 0.0000037 &   6 &    1.0000379 & 0.0000037 \\
        200&     8&  1.0000506 & 0.0000015   &  8 &  1.0000506 & 0.0000050 &   8 &    1.0000506 & 0.0000050 \\
        250&     10& 1.0000632 & 0.0000012   & 10 &  1.0000632 & 0.0000013 &  10 &    1.0000632 & 0.0000063 \\
        300&     13& 1.0000822 & 0.0000012   & 12 &  1.0000759 & 0.0000075 &  12 &    1.0000759 & 0.0000075 \\
        350&     15& 1.0000949 & 0.0000014   & 14 &  1.0000886 & 0.0000088 &  14 &    1.0000886 & 0.0000088 \\
        400&     18& 1.0001139 & 0.0000019   & 16 &  1.0001012 & 0.0000010 &  16 &    1.0001012 & 0.0000010 \\
        450&     20& 1.0001265 & 0.0000026   & 19 &  1.0001202 & 0.0000012 &  19 &    1.0001202 & 0.0000012 \\
        500&     22& 1.0001392 & 0.0000013   & 21 &  1.0001329 & 0.0000013 &  21 &    1.0001329 & 0.0000013 \\
        550&     24& 1.0001519 & 0.0000015   & 23 &  1.0001455 & 0.0000014 &  23 &    1.0001455 & 0.0000014 \\
        600&     26& 1.0001645 & 0.0000016   & 25 &  1.0001582 & 0.0000015 &  25 &    1.0001582 & 0.0000015 \\
        650&     28& 1.0001772 & 0.0000017   & 27 &  1.0001709 & 0.0000017 &  27 &    1.0001709 & 0.0000017 \\
        700&     30& 1.0001898 & 0.0000018   & 29 &  1.0001835 & 0.0000018 &  29 &    1.0001835 & 0.0000018 \\
        750&     33& 1.0002088 & 0.0000020   & 31 &  1.0001962 & 0.0000019 &  31 &    1.0001962 & 0.0000019 \\
        800&     35& 1.0002215 & 0.0000022   & 33 &  1.0002088 & 0.0000020 &  33 &    1.0002088 & 0.0000020 \\
        850&     38& 1.0002405 & 0.0000024   & 36 &  1.0002278 & 0.0000022 &  36 &    1.0002278 & 0.0000022 \\
        900&     39& 1.0002468 & 0.0000024   & 38 &  1.0002405 & 0.0000024 &  38 &    1.0002405 & 0.0000024 \\
        950&     41& 1.0002595 & 0.0000025   & 41 &  1.0002595 & 0.0000025 &  40 &    1.0002531 & 0.0000025 \\
        \bottomrule
    \end{tabular}
    \caption{Die gemessenen Maximadurchläufe des Kontrast wenn die Gaszelle in $\SI{50}{\milli\bar}$ Schritten gefüllt wird. Zu jeder Messung wird der berechnete Brechungsindex der jeweiligen Durchlaufsnummer aufgezeigt.}
    \label{tab:gas_counts}
\end{table}
\end{landscape}
\FloatBarrier
Aus den Messwerten der Tabelle \ref{tab:gas_counts} kann nun der Brechungsindex von Raumluft bestimmt werden.
Dafür werden aus den Anzahl an Maximadurchläufen zunächst die Brechungsindizes von Gas mithilfe von Gleichung \eqref{eq:gasphase} bestimmt.
Diese Brechungsindizes pro Gasdruck werden gegeneinader aufgetragen und sind in Abbildung \ref{fig:n_glass} zu sehen.
Die genauen Zahlenwerte des Brechungsindex zu jeder Durchlaufsanzahl ist zudem in der Tabelle \ref{tab:gas_counts} zu finden.
\begin{figure}
    \centering 
    \includegraphics[width=0.7\textwidth]{build/Gas.pdf}
    \caption{Die berechneten Brechungsindizes von Raumluft werden gegen ihren jeweiligen Druck aufgetragen.
    Es wird eine Ausgleichsrechnung für jede der drei Messreihen durchgeführt.
    Die entsprechenden Fits sind ebenfalls in der Abbildung zu sehen.}
    \label{fig:n_glass}
\end{figure}
Es wird für jede Messreihe ein Ausgleichsrechnung mithilfe \autoref{eq:lorentz} durchgeführt.
Dabei ist $T=\SI{19.2}{\celsius}$ die Raumtemperatur.
Der gesuchte Parameter $A$ wird dabei für jede Messreihe bestimmt, sodass sich folgende Parameter ergeben
\begin{align*}
    A_1 &= \SI{4.468(19)e-6}{\frac{\cubic\meter}{\mol}} \\
    A_2 &= \SI{4.292(20)e-6}{\frac{\cubic\meter}{\mol}} \\
    A_3 &= \SI{4.276(15)e-6}{\frac{\cubic\meter}{\mol}} \, .
\end{align*}
Mithilfe der bestimmten Parameter kann nun der Brechungsindex von Raumluft bei Normaldruck also $\SI{1}{\bar}$ bestimmt werden.
Dazu werden die bestimmten Parameter für $A_i$, sowie der Normaldruck in die Funktion der Ausgleichsrechnung eingesetzt.
So ergeben sich für die drei Messreihen die drei Brechungsindizes für Raumluft
\begin{align*}
    n_\text{Luft, 1} &= \SI{1.0002757(12)}{}\\
    n_\text{Luft, 2} &= \SI{1.0002648(12)}{}\\
    n_\text{Luft, 3} &= \SI{1.0002638(09)}{}
\end{align*}
aus welchen der Mittelwert gebildet wird.
So ergibt sich im Mittel der drei Messungen der Brechungsindex
\begin{align*}
    \bar{n}_\text{Luft} &= \SI{1.0002681(07)}{}
\end{align*}
für die Raumluft bei Normaldruck und einer Temperatur von $T=\SI{19.2}{\celsius}$.
Anschließend wird der Mittelwert des Parameters $A$ genutzt um mithilfe von Gleichung \eqref{eq:lorentz} der Brechungsindex für Normatmosphäre zu berechnen.
Es werden also für den Druck $p=\SI{101325}{\Pa}$ und die Temperatur $T=\SI{15}{\celsius}$ eingesetzt.
So ergibt sich ein Brechungsindex von 
\begin{align*}
    n_\text{Luft, Normal} = \SI{1.0002756(7)}{}
\end{align*}
für Luft bei Normatmosphäre.
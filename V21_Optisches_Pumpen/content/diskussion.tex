\newpage
\section{Diskussion}
\label{sec:diskussion}
\subsection{Magnetfeld der Erde}
Zur Komepensation des Erdmagnetfeld in vertikaler Richtung wird ein Magnetfeld der Stärke $B_v = \SI{3.45e-5}{\tesla}$ angelegt.
Dieses weicht von dem tatsächlich in Dortmund herschenden Magnetfeld von $B_v=\SI{4.52e-5}{\tesla}$ \cite{mag_DO_2} um ungefähr $20\%$ ab.
Der Grund dafür können lokale Schwankungen des Erdmagnetfelds sein oder der Einfluss anderer Experiment die ebenfalls Magnetfelder nutzen und im selben Raum stehen.
Die genutzten B-Felder zur Komepensation des Erdmagnetfelds in horizontaler Richtung beläufen sich auf die Werte von
\begin{align*}
b_\text{Iso1} & = \SI{2.23(13)e-05}{\tesla} \\
b_\text{Iso2} &= \SI{2.34(7)e-05}{\tesla} \, .
\end{align*}
Hier beträgt der in Dortmund herrschende Wert $B_h=\SI{19.34}{\micro\tesla}$ was einer Abweichung von ungefähr $21\%$ entspricht.
\subsection{Kernspin}
Die aus den Landefaktoren bestimmten Kernspins 
\begin{align*}
    I_\text{Iso1} &= \SI{1.522(30)}{}\\
    I_\text{Iso2} &= \SI{2.515(15)}{}
\end{align*}
weichen um unter $1.5\%$ von ihren theoretischen Werten ab.
Damit scheint dieser Versuch ein zuverlässiges Ergebnisse für die Kernspins der beiden genutzten Isotope zu liefern.
\subsection{Isotopenverhältnis}
Das Isotopenverhältnis von 
\begin{align*}
    ^{87}\text{Rb} \, &\widehat{=}\, 34.07\% \\
    ^{85}\text{Rb} \, &\widehat{=}\, 65.92\%
\end{align*}
weicht von dem natürlichen Verhältnisse
\begin{align*}
    ^{87}\text{Rb} \, &\widehat{=}\, 28\% \\
    ^{85}\text{Rb} \, &\widehat{=}\, 72\%
\end{align*}
ab was mit der schlechten Auflösung der Aufnahme des Bildes begründet werden kann.
In Zukunft sollte darauf geachtet werden, die $x-Achse$ des Oszilloskop bei diesem Versuchsteil so kleinschrittig wie möglich zu wählen.
\subsection{Rabioszillation}
Der zu bestimmende Quotient der $b-$Parametern wird auf ein Wert von
\begin{align*}
    \frac{b_\text{Iso2}}{b_\text{Iso1}} &= \SI{1.525(26)}{} \, .
\end{align*}
bestimmt.
Nach der Versuchsanleitung ist der theoretisch Wert hierfür $1.5$. 
Damit ist die Abweichung des gemessenen Wert zum Theoriewert $\SI{1.6(1.7)}{}\%$.
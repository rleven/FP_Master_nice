\newpage
\section{Diskussion}
\label{sec:diskussion}

\subsection{Halbwertsbreite der Verzögerung}
Als Pulsdauer wurden $\Delta t = \SI{15}{\nano\second}$ angelegt.
Da beide Verzögerungsleitungen symmetrisch genutzt wurden, also von \SI{-15}{\nano\second} bis \SI{15}{\nano\second}, wird eine Halbwertsbreite von \SI{30}{\nano\second} erwartet.\\
Die gemessene Halbwertsbreite weicht mit ihrem Wert von $fwhm = \left(29.305\pm0.0014\right)\si{\nano\second}$ $\left(2.317\pm0.005\right)\%$ davon ab.
Zu erklären ist diese Abweichung mit möglichen Defekten der Verzögerungsleitungen, wie beispielsweise Wackelkontakten, die bei der Durchführung beobachtet wurden, und die Einbrüche in der Messung zu \autoref{fig:fwhm} verursacht haben könnten.

\subsection{Lebensdauer von Myonen}
Der Literaturwert zur Lebensdauer von Myonen liegt bei $\tau_{\mu} = \SI{2.196}{\micro\second}$, sodass der gemessene Wert von $\tau = \left(2.5454 \pm 0.0417\right)\si{\micro\second}$ einen Unterschied von $\left(13.7267\pm1.4134\right)\%$ aufweist.
Dieser Wertunterschied kann unter anderem vermindert werden durch einen größeren Versuchsaufbau mit höherem Szintillatortankvolumen und einer längeren Messzeit.

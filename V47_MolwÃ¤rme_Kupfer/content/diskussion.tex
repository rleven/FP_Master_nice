\newpage
\section{Discussion}
\label{sec:diskussion}

The average Debye temperature has been calculated to be $\theta_D \approx 415.425\si{\kelvin}$.
The literature value $\theta_D = 345\si{\kelvin}$ implies a divergence of around \SI{16.95}{\percent}.
The theoretically calculated value differs from measurement and literature, being $\theta_D \approx 332.208\si{\kelvin}$.
Compared to measurement its \SI{20.03}{\percent} smaller.\\
This concludes that the measured average value was influenced by exterior factors.
Those could have been a weak vacuum in the probe chamber or a malfunction of the corresponding pressure gauge.
Additionally the discrepancy between shield temperature and probe temperature has been kept at an average of \SI{1.89}{\kelvin}, instead of the recommended 7 to 11 \si{\kelvin}.\\
There are noticeable jumps in \autoref{fig:cp} and \autoref{fig:cv} at around \SI{180}{\kelvin}, which influenced the outcome of $\theta_D$.
They were caused by increasing the voltage in order to speed up the heating process to save experimentation time.
As the errors of the measurements were relatively small, see for example \autoref{fig:temp}, the outcome is not strongly dependent on that.\\
It is noteworthy that due to the Debye number being determined only in steps of $0.1$ an additional increase in difference between theory and experiment is to be expected.\\
\newline
To sum up the molar heat capacity of copper was measured to be approximately $22.5\frac{J}{Kmol}$ with $C_V$ being slightly less as expected.
That is a difference of \SI{9.99}{\percent} to the literature value of $C = 25\frac{J}{Kmol}$\cite{copper2}.
The Debye temperature difference lies at around \SI{17}{\percent} to \SI{20}{\percent} depending on theoretical or literature value.